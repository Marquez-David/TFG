
\chapter{Conclusiones y líneas futuras}
\label{cha:concl-y-line}

En este apartado se resumen las conclusiones obtenidas y se proponen
futuras líneas de investigación que se deriven del trabajo.

La estructura del capítulo es...


\section{Conclusiones}
\label{sec:conclusiones}

Para añadir una referencia a un autor, se puede utilizar el paquete
\texttt{cite}. En el trabajo \cite{armani03}, se muestra un trabajo...

Y podemos usar de nuevo algún acrónimo, como por ejemplo \ac{TDPSOLA}, o
uno ya referenciado como \ac{ANN}.


\section{Líneas futuras}
\label{sec:lineas-futuras}

Pues eso.


%%% Local Variables:
%%% TeX-master: "../book"
%%% End:


