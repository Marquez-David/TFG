\chapter*{Resumen}
\label{cha:resumen}
\markboth{Resumen}{Resumen}

%\addcontentsline{toc}{chapter}{Resumen}

El minado web surge como solución al costoso y tradicional proceso de extracción. Este nuevo procedimiento 
es adoptado por múltiples empresas e instituciones con el objetivo de automatizar y optimizar el proceso 
de obtención de información. ¿Qué herramientas existen en el mercado y como funcionan? ¿Qué diferencias
existen con otras ya existentes?. El objetivo del trabajo es realizar un análisis profundo de los paquetes
de programación más importantes dentro del \emph{web scraping}, con el objetivo posterior de efectuar una
evaluación cuantitativa de los mismos.

En respuesta a todas estas preguntas, se inicia un proceso de estudio. En el mismo se confecciona un método
de evaluación basado, por un lado, en la calidad de la información extraída de cada herramienta, y por 
otro en el empleo de recursos para cada extracción. La cantidad y el contenido \emph{boilerplate} de la 
información extraída, así como el uso de CPU/RAM o el tiempo de ejecución serán variables sometidas a prueba.

Los resultados obtenidos reflejan que aquellos algoritmos cuya heurística es más sofisticada suelen
presentar mejores soluciones. Se ha detectado, además, que tanto el lenguaje, como el analizador empleado
tienen una importancia menor con respecto a la eficacia de la extracción, pero no en la eficiencia. Como
conclusión se puede determinar que tanto el objetivo del paquete como su implementación, son fundamentales
para lograr correctos resultados.

\textbf{Palabras clave:} \mybookpalabrasclave.