\chapter{Introducción y objetivos}
\label{cha:introduccion y objetivos}

\section{Contexto}
\label{sec:contexto}

La \textit{World Wide Web} o lo que comúnmente se conoce como la web, es la estructura de datos más
grande en la actualidad, y continúa creciendo de forma exponencial. Este gran crecimiento se debe a que el
proceso de publicación de dicha información se ha ido facilitando con el tiempo.

Tradicionalmente el proceso de inserción y extracción de la información se realizaba a través del
\emph{copy-paste}. Aunque este método en ocasiones pueda ser la única opción, es una técnica muy ineficiente 
y poco productiva, pues provoca que el conjunto final de datos no esté bien estructurado. El \emph{web 
scraping} o minado web trata precisamente de eso, de automatizar la extracción y almacenamiento de 
información extraída de un sitio web \cite{Andreas-Mehlfuhrer}.

La forma en la que se extraen datos de internet puede ser muy diversa, aunque comúnmente se emplea el
protocolo HTTP, existen otras formas de extraer datos de una web de forma automática
\cite{web-scraping-bozhao}. Este proyecto, se centra en la metodología existente de obtención de
información, de como se tratan los datos y la forma en la que se almacenan. Durante los siguientes
apartados se realizará una especificación mas concreta del objetivo del proyecto, así como de la
estructura y limitaciones del mismo.

\section{Motivación}
\label{sec:motivacion}

El proceso de extracción y recopilación de datos no estructurados en la web es un área interesante en
muchos contextos, ya sea para uso científico o personal. En ciencia por ejemplo, los conjuntos de datos
se comparten y utilizan por múltiples investigadores, y a menudo también  son compartidos públicamente.
Dichos conjuntos de datos se proporcionan a través de una API \footnote{Conjunto de funciones
y procedimientos que ofrece cierta biblioteca para ser utilizada por otro software como abstracción.} 
estructurada, pero puede suceder que solo sea posible acceder a ellos a través de formularios de búsqueda 
y documentos HTML. En el uso personal también ha crecido a medida que han comenzado a surgir servicios que 
proporcionan a los usuarios herramientas para combinar información de diferentes sitios web en su propias 
colección de páginas.

Además de ser un ámbito interesante, el minado web también es un área muy requerida, algunos de los campos
de mayor demanda tienen relación con la venta minorista, mercado de valores, análisis de las redes sociales,
investigaciones biomédicas, psicología...

\section{Objetivo y limitaciones}
\label{sec:objetivo y limitaciones}

Existen muchos tipos de técnicas y herramientas para realizar \emph{web scraping}, desde programas con 
interfaz gráfica, hasta bibliotecas software de desarrollo. El objetivo de esta tesis es realizar un análisis
cuantitativo de las diferentes técnicas y paquetes software para el desarrollo de \emph{web scraping}.

¿Cuál es la solución más rentable para el minado web? ¿Cuál de las soluciones tiene un mejor rendimiento?
Para responder a esta pregunta, se realizará un estudio comparando las diferentes características de los
paquetes software, con el objetivo finalmente de poder determinar cuál es el más óptimo en términos de
memoria, rendimiento...

En cuanto a las restricciones para el funcionamiento de un \emph{scraper}, estas pueden ser varias, ya sean 
legales o por la incapacidad de acceder a una gran parte del contenido no indexado en internet. Aunque el 
uso de los \emph{scrapers} está generalmente permitido, en algunos países como en Estados Unidos, las 
cortes en múltiples ocasiones han reconocido que ciertos usos no deberían estar autorizados 
\cite{Andreas-Mehlfuhrer}. El desarrollo de este proyecto no se verá perjudicado por este tipo de 
cuestiones, pues solo se limitará al estudio y análisis de los mismos.

\section{Estructura del documento}
\label{sec:estructura del documento}

Para poder facilitar la composición de la memoria se detalla a continuación la estructura de la misma:
\begin{enumerate}
  \item {\bfseries Bloque I: Introducción. }
        \begin{itemize}
          \item {\bfseries Capítulo 1: Introducción. \justify}
                En la introducción se especifica tanto el contexto como la motivación a realizar el 
                proyecto, así como las limitaciones esperadas durante la realización del mismo.
        \end{itemize}
  \item {\bfseries Bloque II: Marco teórico. }
        \begin{itemize}
          \item {\bfseries Capítulo 2: Web scraping, extracción de datos en la web. \justify}
                Durante este capítulo se explica en que consiste el \emph{web scraping}, sus posibilidades 
                prácticas y aspectos más generales.
          \item {\bfseries Capítulo 3: Introducción a los paquetes seleccionados y proceso de búsqueda. \justify}
                Se realiza la selección de paquetes y se dictamina cuál ha sido la razón por la que los 
                paquetes han sido seleccionados. Además, se especifican las características principales 
                de cada uno, así como una visión general de sus funcionalidades.
        \end{itemize}
  \item {\bfseries Bloque III: Marco práctico. }
        \begin{itemize}
          \item {\bfseries Capítulo 4: Selección de variables de análisis y proceso de estudio. \justify}
                Durante este capítulo se especifica el proceso de análisis a realizar, cuáles son los test 
                a los que los paquetes serán sometidos y que variables se tomarán a estudio para los 
                mismos.
          \item {\bfseries Capítulo 5: Análisis y comparativa de paquetes. \justify}
                Una vez introducidos todos los paquetes y el estudio al que van a ser sometidos, se realizá 
                la comparativa de los mismos. Inicialmente, los paquetes serán analizados uno por uno y 
                finalmente se hará una comparativa con los datos obtenidos.
        \end{itemize}
  \item {\bfseries Bloque IV: Conclusiones y futuras líneas de trabajo. }
\end{enumerate}

