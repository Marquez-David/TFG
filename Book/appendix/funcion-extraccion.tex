
\chapter{Funciones implementadas}
\label{cha:funciones}

\section{Función de extracción de Tweets}
\label{sec:funcion-extraccion}

\begin{codefloat}
\inputencoding{latin1}
\lstinputlisting[style=CppExample]{appendix/function.c}
\inputencoding{utf8}
\caption{Extracción de Tweets empleando una funcion en R}
\label{cod:extraccion}
\end{codefloat}

El objetivo fundamental de esta función es extraer los tweets publicados por un usuario y almacenarlos 
en un archivo csv. En caso de que exista un archivo con el mismo nombre, se lee y se concatena el 
nuevo contenido con el antiguo. Los argumentos de entrada, son los siguientes:

\begin{enumerate}

\item usuario: reprsenta el identificador del usuario de Twitter.
\item maxtweets: cantidad de tweets que se van a recuperar.
\item archivoSalida: nombre del fichero de salida.

\end{enumerate}

Se debe tener cuidado de no recuperar tweets repetidos, para ello se ha creado una variable que almacena 
el id del ultimo tweet. Con esto cada vez que se quieran recuperar nuevos tweets, se aumentará en 
uno dicha variable y se procederá como hasta ahora.

%%% Local Variables:
%%% TeX-master: "../book"
%%% End:


