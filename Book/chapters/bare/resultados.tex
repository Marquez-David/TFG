
\chapter{Resultados}
\label{cha:resultados}


\begin{FraseCelebre}
  \begin{Frase}
    % Si quieres ser leído más de una vez, no vaciles en borrar a menudo.
    Rem tene, verba sequentur (Si dominas el tema, las palabras vendrán
    solas)\footnote{Tomado de ejemplos del proyecto \texis{}.}.
  \end{Frase}
  \begin{Fuente}
    % Horacio
    Catón el Viejo
  \end{Fuente}
\end{FraseCelebre}

\section{Introducción}
\label{sec:results-introduction}

Blah, blah, blah.

La estructura de este capítulo es\ldots


\section{Sección 1 del capítulo de resultados}
\label{sec:results-1}

Blah, blah, blah.


\subsection{Subsección 1.1 del capítulo de resultados}
\label{sec:results-11}

Blah, blah, blah.


\subsection{Subsección 1.2 del capítulo de resultados}
\label{sec:results-12}

Blah, blah, blah.




\section{Sección 2 del capítulo de resultados}
\label{sec:results-2}

Blah, blah, blah.




\section{Conclusiones}
\label{sec:results-conclusions}

Blah, blah, blah.


%%% Local Variables:
%%% TeX-master: "../book"
%%% End:
