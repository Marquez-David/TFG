\chapter*{Abstract}
\label{cha:abstract}

%\addcontentsline{toc}{chapter}{Abstract}

Web mining has emerged as a solution to the costly and traditional mining process. This new procedure is 
adopted by many companies and institutions with the aim of automating and optimizing the process of obtaining 
information. What tools are available on the market and how do they work? What are the differences with 
other existing tools? What are the differences with other existing ones? The objective of this work is to 
carry out an in-depth analysis of the important programming packages within web scraping, with the 
subsequent objective of making a quantitative evaluation of them.

In response to all these questions, a study process is initiated. In this process, an evaluation method is
evaluation method based, on the one hand, on the quality of the information extracted from each tool,
on the other hand, on the use of resources for each extraction. The quantity and boilerplate content of 
the information extracted, as well as CPU/RAM usage or execution time will be variables tested.

The results obtained show that those algorithms whose heuristics are more sophisticated tend to present 
better solutions. It has also been found that both the language and the parser used are of lesser importance 
with respect to the heuristics used. It has also been found that both the language and the analyzer used 
are of minor importance with respect to the effectiveness of the extraction, but not in the efficiency. In 
conclusion, it can be determined that both the objective of the package and its implementation are 
fundamental to achieve correct results.

\textbf{Keywords:} \mybookkeywords.