\chapter{Paquetes orientados al minado web en Python}
\label{cha:paquetes oprientados al minado web en python}

Se realiza tanto un proceso de búsqueda, como un proceso de selección de herramientas capaces de realizar 
web scraping. Como ya se mencionó en la sección \ref{subsec:herramientas software disponibles} este trabajo 
se centra sobre bibliotecas de programación, por lo que tanto frameworks como entornos de escritorio 
quedaran exentos análisis y posteriores test. Con el fin de encontrar la mayor cantidad de paquetes 
posibles se opta por emplear los siguientes repositorios:

\begin{itemize}
  \item GitHub.
  \item PyPI.
  \item Libreria estandar de Python.
\end{itemize}

A lo largo de las siguientes secciones se mostrarán las librerías de código abierto más interesantes para
entre propósito, además se dará una breve explicación del funcionamiento y peculiaridades de las mismas.

\section{Proceso de búsqueda y selección de paquetes orientados al web scraping}
\label{sec:proceso de busqueda y seleccion de paquetes orientados al web scraping}


\section{Paquetes encontrados durante el proceso de búsqueda}
\label{sec:paquetes encontrados durante el proceso de busqueda}

\section{Paquetes seleccionados para el proceso de análisis}
\label{sec:paquetes seleccionados para el proceso de analisis}


















