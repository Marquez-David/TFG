%%%%%%%%%%%%%%%%%%%%%%%%%%%%%%%%%%%%%%%%%%%%%%%%%%%%%%%%%%%%%%%%%%%%%%%%%%% 
% 
% Generic template for the anteproyectos of TFC/TFM/TFGs
% 
% $Id: anteproyecto.tex,v 1.6 2018/09/11 12:23:48 macias Exp $
% 
% By:
%  + Javier Macías-Guarasa. 
%    Departamento de Electrónica
%    Universidad de Alcalá
%  + Roberto Barra-Chicote. 
%    Departamento de Ingeniería Electrónica
%    Universidad Politécnica de Madrid   
% 
% Based on original sources by Roberto Barra, Manuel Ocaña, Jesús Nuevo,
% Pedro Revenga, Fernando Herránz and Noelia Hernández. Thanks a lot to
% all of them, and to the many anonymous contributors found (thanks to
% google) that provided help in setting all this up.
% 
% See also the additionalContributors.txt file to check the name of
% additional contributors to this work.
% 
% If you think you can add pieces of relevant/useful examples,
% improvements, please contact us at (macias@depeca.uah.es)
% 
% You can freely use this template and please contribute with
% comments or suggestions!!!
% 
%%%%%%%%%%%%%%%%%%%%%%%%%%%%%%%%%%%%%%%%%%%%%%%%%%%%%%%%%%%%%%%%%%%%%%%%%%% 

% This is for rubber to clean additional files
% rubber: clean anteproyecto.acn anteproyecto.acr anteproyecto.alg anteproyecto.cod anteproyecto.ist anteproyecto.out anteproyecto.sbl anteproyecto.slg anteproyecto.sym anteproyecto.lor

%%%%%%%%%%%%%%%%%%%%%%%%%%%%%%%%%%%%%%%%%%%%%%%%%%%%%%%%%%%%%%%%%%%%%%%%%%% 
% BEGIN Preamble and configuration section
% 
\input{../Config/preamble-anteproyecto.tex}    % DO NOT TOUCH THIS LINE. You can edit
% the file to modify some default settings

% Control language specific modifications
% This can be english or spanish
\newcommand{\mybooklanguage}{spanish}
%\newcommand{\mybooklanguage}{english}

% Control compilation flavour (for PFCs, TFMs, TFGs, Thesis, etc...)
% Degree (titulación), can be:
% IT     - Ingeniería de Telecomunicación
% IE     - Ingeniería Electrónica
% ITTSE  - Ingeniería Técnica de Telecomunicación, Sistemas Electrónicos
% ITTST  - Ingeniería Técnica de Telecomunicación, Sistemas de Telecomunicación
% ITI    - Ingeniería Técnica Industrial, Electrónica Industrial 
% GITT   - Grado en Ingeniería en Tecnologías de la Telecomunicación
% GIEC   - Grado en Ingeniería Electrónica de Comunicaciones
% GIT    - Grado en Ingeniería Telemática
% GIST   - Grado en Ingeniería en Sistemas de Telecomunicación
% GIC    - Grado en Ingeniería de Computadores
% GII    - Grado en Ingeniería Informática
% GSI    - Grado en Sistemas de Información
% GISI   - Grado en Ingeniería en Sistemas de Información
% GIEAI  - Grado en Ingeniería en Electrónica y Automática Industrial
% GITI   - Grado en Ingeniería en Tecnologías Industriales
% MUSEA  - Máster Universitario en Sistemas Electrónicos Avanzados. Sistemas Inteligentes
% MUIT   - Máster Universitario en Ingeniería de Telecomunicación
% MUII   - Máster Universitario en Ingeniería Industrial
% PHDUAH - Doctorado UAH
% PHDUPM - Doctorado UPM
% GEINTRARR - Geintra Research Report (alpha support)
% You can include additional degrees and modify config/myconfig.tex
% config/postamble.tex and cover/cover.tex, generating new specific
% cover files if needed
\newcommand{\mydegree}{GII}
%\newcommand{\mydegree}{PHDUAH}

\newcommand{\mybookSplittedAdvisors}{true} % if false it will set
                                % "Tutores/Advisors" in the cover
                                % pages. Otherwise it will split in
                                % Tutor/Cotutor Advisor/Co-advisor


\newcommand{\myspecialty}{} % New in TFGs from 20151218!

% General document information
\newcommand{\mybooktitlespanish}{Análisis y comparacion de paquetes para el desarrollo de web scraping en R}
\newcommand{\mybooktitleenglish}{Analysis and packages comparison for webscraping development in R}

\newcommand{\mybookauthorname}{David}
\newcommand{\mybookauthorsurname}{Márquez Mínguez}
\newcommand{\mybookauthor}{\mybookauthorname{} \mybookauthorsurname{}}
%\newcommand{\mybookauthorgender}{female}
\newcommand{\mybookauthorgender}{male}

\newcommand{\mybookdepartment}{Departamento de Ciencias de la Computación}
\newcommand{\mybookdepartmentEnglish}{Computer Science Department}
\newcommand{\mybookphdprogram}{}
\newcommand{\mybookphdprogramEnglish}{}
\newcommand{\mybookresearchgroup}{}
\newcommand{\mybookschool}{Escuela Politécnica Superior}
\newcommand{\mybookuniversity}{Universidad de Alcalá}
\newcommand{\mybookuniversityacronym}{UAH}
\newcommand{\mybookauthordegree}{Ingeniero Informático} % Used in UPM
\newcommand{\mybookemail}{david.marquez@edu.uah.es}

\newcommand{\mybookNameAcademicTutor}{Juan José Cuadrado Gallego} % This is the default in  TFGs from 20151218
\newcommand{\mybookAcademicTutorGender}{male}
\newcommand{\mybookNameCoTutor}{} % In case you need this for yout TF?
\newcommand{\mybookCoTutorGender}{}

\newcommand{\mybookNameFirstAdvisor}{\mybookNameAcademicTutor} % This is deprecated: set to academic tutor
\newcommand{\mybookNameSecondAdvisor}{\mybookNameCoTutor} % This is deprecated: set to cotutor
\newcommand{\mybookpresident}{. . . . . . . . . . . . . . . . . . . . . . . . . . . . . .}
\newcommand{\mybookfirstvocal}{. . . . . . . . . . . . . . . . . . . . . . . . . . . . . .}
\newcommand{\mybooksecondvocal}{. . . . . . . . . . . . . . . . . . . . . . . . . . . . . .} % At UAH usually \mybookNameFirstAdvisor
\newcommand{\mybookalternatemember}{Name of the alternate member}
\newcommand{\mybooksecretary}{Name of the secretary (if needed)}

% Calendar dates    
\newcommand{\mybookyear}{2021}

\newcommand{\myanteproyectodate}{26 de noviembre de 2020}

\newcommand{\mydepositdate}{ . . . . de . . . . .  de . . . . .} % The date you deposit (submit) this document in the Department
\newcommand{\mydepositdateEnglish}{X X\textsuperscript{st}, 2021} 

% For RR, mydefensedate is date to be shown in the cover
\newcommand{\mydefensedate}{6 de enero de 2018}
\newcommand{\mydefensedateEnglish}{January 6\textsuperscript{th}, 2018}
% If you prefer British English for the date, use this:
% \newcommand{\mydefensedateEnglish}{6\textsuperscript{th} of January, 2018}

\newcommand{\mybookkeywords}{Bachelor final project , \LaTeX, English/Spanish support, maximum of five...} % (up to a maximum of five)
\newcommand{\mybookpalabrasclave}{Trabajo fin de /grado, \LaTeX, soporte de español e inglés, hasta cinco...} % (máximo de cinco)

%\newcommand{\mybookvicerrectorinvestigacion}{Excma. Sra. María Luisa Marina Alegre}
\newcommand{\mybookvicerrectorinvestigacion}{Excmo. Sr. Francisco J. de la Mata de la Mata}
% Por TFGs & TFMs & MUSEA-TFMs paperwork
\newcommand{\mybookdepartmentsecretary}{José Luis Martín Sánchez}
\newcommand{\mybookdateforpaperwork}{22 de mayo de 2019}
\newcommand{\mybookDNIOpenPublishing}{47319570-Z} % Required for TFG's & MUSEA-TFMs
                                % paperwork, must be the DNI of the student
\newcommand{\mybookDNIAcademicTutor}{11111111-A}
\newcommand{\mybookDNICotutor}{}
\newcommand{\mybookDNIFirstAdvisor}{\mybookDNIAcademicTutor} % Deprecated: set to that of academic tutor
\newcommand{\mybookDNISecondAdvisor}{\mybookDNICotutor} % Deprecated set to that of cotutor
\newcommand{\mybookFigure}{alumno} % Required
                                % for TFG's: the type of adscription of
                                % the author signing the agreement
                                % (should be "alumno" in most cases)

\newcommand{\mybookresearchreportID}{RR-2018-01}

% Personal details for the anteproyecto request
% Not required in some cases
\newcommand{\mystreet}{C/Calle de la Ilustración n3}
\newcommand{\mycity}{Getafe}
\newcommand{\mypostalcode}{28903}
\newcommand{\myprovince}{Madrid}
\newcommand{\mytelephone}{689770454}


% Link color definition
% Color links of the toc/lot/lof entries
%\newcommand{\mytoclinkcolor}{blue}
\newcommand{\mytoclinkcolor}{black}
%\newcommand{\myloflinkcolor}{red}
\newcommand{\myloflinkcolor}{black}
%\newcommand{\mylotlinkcolor}{green}
\newcommand{\mylotlinkcolor}{black}

% This is used in cover/extralistings.tex
%\newcommand{\myothertoclinkcolor}{magenta}
\newcommand{\myothertoclinkcolor}{black}

% Other color links in the document
\newcommand{\mylinkcolor}{blue}
%\newcommand{\mylinkcolor}{black}

% Color links to urls and cites
\newcommand{\myurlcolor}{blue}
%\newcommand{\myurlcolor}{black}
\newcommand{\mycitecolor}{blue}
%\newcommand{\mycitecolor}{black}

% END Set my own variables (control compilation for different flavours)
%%%%%%%%%%%%%%%%%%%%%%%%%%%%%%%%%%%%%%%%%%%%%%%%%%%%%%%%%%%%%%%%%%%%%%%%%%% 

%%%%%%%%%%%%%%%%%%%%%%%%%%%%%%%%%%%%%%%%%%%%%%%%%%%%%%%%%%%%%%%%%%%%%%%%%%% 
% BEGIN My bibliography database files
% Define your own commands here

% This should be relative to the path in which book.tex is located
\newcommand{\myreferences}{biblio/biblio}

% END My bibliography database files
%%%%%%%%%%%%%%%%%%%%%%%%%%%%%%%%%%%%%%%%%%%%%%%%%%%%%%%%%%%%%%%%%%%%%%%%%%% 

%%%%%%%%%%%%%%%%%%%%%%%%%%%%%%%%%%%%%%%%%%%%%%%%%%%%%%%%%%%%%%%%%%%%%%%%%%% 
% BEGIN My own commands section 
% Define your own commands here

% This one is to define a specific format for english text in a Spanish
% document
\DeclareRobustCommand{\texten}[1]{\textit{#1}}

\def\ci{\perp\!\!\!\perp}

% Various examples of commonly used commands
\newcommand{\circulo}{\large $\circ$}
\newcommand{\asterisco}{$\ast$}
\newcommand{\cuadrado}{\tiny $\square$}
\newcommand{\triangulo}{\scriptsize $\vartriangle$}
\newcommand{\triangv}{\scriptsize $\triangledown$}
\newcommand{\diamante}{\large $\diamond$}

\newcommand{\new}[1]{\textcolor{magenta}{#1 }}
\newcommand{\argmax}[1]{\underset{#1}{\operatorname{argmax}}}

% This is an example used in the sample chapters
\newcommand{\verticalSpacingSRPMaps}{-0.3cm}

% END My own commands section 
%%%%%%%%%%%%%%%%%%%%%%%%%%%%%%%%%%%%%%%%%%%%%%%%%%%%%%%%%%%%%%%%%%%%%%%%%%% 

%%% Local Variables:
%%% TeX-master: "../book"
%%% End:


    % DO NOT TOUCH THIS LINE, but EDIT THIS FILE 
                                  % to set your specific settings (related
                                  % to the document language, your degree,
                                  % document details (such as title, author
                                  % (you), your email, name of the tribunal
                                  % members, document year, keyword and
                                  % palabras clave) and link colors), and
                                  % define your commonly used commands
                                  % (some examples are provided).

\input{../Config/postamble.tex}   % DO NOT TOUCH THIS LINE. Yes, I know,
                                  % "postamble" is not a valid word... :-)

% path to directories containing images
\graphicspath{{../Book/logos/}{../Book/figures/}{../Book/diagrams/}{../Book/photos/}} % Edit this to your
                                  % needs. Only logos is really required
                                  % when you generate your own content.
% 
% END Preamble and configuration section
%%%%%%%%%%%%%%%%%%%%%%%%%%%%%%%%%%%%%%%%%%%%%%%%%%%%%%%%%%%%%%%%%%%%%%%%%%% 

\title{Anteproyecto de \mybookworktypefull}        % DO NOT TOUCH THIS LINE
\date{\myanteproyectodate}                         % DO NOT TOUCH THIS LINE
\author{\mybookauthor}

%%%%%%%%%%%%%%%%%%%%%%%%%%%%%%%%%%%%%%%%%%%%%%%%%%%%%%%%%%%%%%%%%%%%%%%%%%% 
% Let's start with the real stuff
%%%%%%%%%%%%%%%%%%%%%%%%%%%%%%%%%%%%%%%%%%%%%%%%%%%%%%%%%%%%%%%%%%%%%%%%%%% 
\begin{document}                                   % DO NOT TOUCH THIS LINE

%%%%%%%%%%%%%%%%%%%%%%%%%%%%%%%%%%%%%%%%%%%%%%%%%%%%%%%%%%%%%%%%%%%%%%%%%%% 
% BEGIN within-document configuration, frontpage and cover pages generation
% 

% Set Language dependent issues that must be set after \begin{document}


\ifthenelse{\equal{\mybooklanguage}{english}}
{
  \selectlanguage{english}

  \newcommand{\xUseSpanish}{~}    
  \newcommand{\xUseEnglish}{X}  

  \floatname{codefloat}{Listing}
  \renewcommand{\lstlistingname}{Listing}

  %% \DeclareFloatingEnvironment[
  %%       fileext=lox,
  %%       listname={Source code listing},
  %%       name=Code,
  %%       placement=p,
  %%       within=section,
  %%       chapterlistsgaps=off,
  %%       ]{sourcecode}
}
{
  \selectlanguage{spanish}

  \newcommand{\xUseSpanish}{X}    
  \newcommand{\xUseEnglish}{~}  
  \renewcommand{\tablename}{Tabla}
  \renewcommand{\listtablename}{Índice de tablas}

  \floatname{codefloat}{Listado}
  \renewcommand{\lstlistingname}{Listado}


  %% \DeclareFloatingEnvironment[
  %%       fileext=lox,
  %%       listname={Lista de código fuente},
  %%       name=Código,
  %%       placement=p,
  %%       within=section,
  %%       chapterlistsgaps=off,
  %%       ]{sourcecode}
}

%%% Local Variables:
%%% TeX-master: "../book"
%%% End:


 % DO NOT TOUCH THIS LINE
                                                 % NOR THE FILE

% 
% END within-document configuration, frontpage and cover pages generation
%%%%%%%%%%%%%%%%%%%%%%%%%%%%%%%%%%%%%%%%%%%%%%%%%%%%%%%%%%%%%%%%%%%%%%%%%%% 

\maketitle

\begin{description}                               % DO NOT TOUCH THIS LINE
\item[Título:] \mybooktitlespanish                % DO NOT TOUCH THIS LINE
  \ifthenelse{\equal{\mybooklanguage}{english}}   % DO NOT TOUCH THIS LINE
  {                                               % DO NOT TOUCH THIS LINE
  \item[Título en inglés:] \mybooktitleenglish    % DO NOT TOUCH THIS LINE
  }                                               % DO NOT TOUCH THIS LINE
  {                                               % DO NOT TOUCH THIS LINE
  }                                               % DO NOT TOUCH THIS LINE
\item[Departamento:] \mybookdepartment            % DO NOT TOUCH THIS LINE
\item[\expandafter\makefirstuc\expandafter{\mybookAutorOrAutora}:] \mybookauthor                       % DO NOT TOUCH THIS LINE
\item[\expandafter\makefirstuc\expandafter{\mybookTutorOrTutores}:] \mybookadvisors                   % DO NOT TOUCH THIS LINE
\end{description}                                 % DO NOT TOUCH THIS LINE

%%%%%%%%%%%%%%%%%%%%%%%%%%%%%%%%%%%%%%%%%%%%%%%%%%%%%%%%%%%%%%%%%%%%%%%%%%% 
%%%%%%%%%%%%%%%%%%%%%%%%%%%%%%%%%%%%%%%%%%%%%%%%%%%%%%%%%%%%%%%%%%%%%%%%%%% 
%%%%%%%%%%%%%%%%%%%%%%%%%%%%%%%%%%%%%%%%%%%%%%%%%%%%%%%%%%%%%%%%%%%%%%%%%%% 
%%%%%%%%%%%%%%%%%%%%%%%%%%%%%%%%%%%%%%%%%%%%%%%%%%%%%%%%%%%%%%%%%%%%%%%%%%% 
%%%%%%%%%%%%%%%%%%%%%%%%%%%%%%%%%%%%%%%%%%%%%%%%%%%%%%%%%%%%%%%%%%%%%%%%%%% 
%%%%%%%%%%%%%%%%%%%%%%%%%%%%%%%%%%%%%%%%%%%%%%%%%%%%%%%%%%%%%%%%%%%%%%%%%%% 
%%%%%%%%%%%%%%%%%%%%%%%%%%%%%%%%%%%%%%%%%%%%%%%%%%%%%%%%%%%%%%%%%%%%%%%%%%% 
% BEGIN Normal sections. Edit/modify all within this section

\section{Introducción}
\label{sec:introduccion}

El análisis de datos en particular y la ciencia de los datos de forma general, son  disciplinas con múltiples 
facetas y enfoques, no solo en el campo informático sino en multitud de ambitos ya que permite 
obtener información a partir de una serie de datos.

En este trabajo la atención se centrará en la extracción de información a partir de fragmentos de texto.
A diferencia de lo que comúnmente se conoce como minería de datos, en la minería de texto la información 
no se obtiene directamente a partir de los datos, sino que esta información se obtiene a partir de grandes 
cantidades de texto. Una vez extraida, la información no suelen estar ni estructurada ni limpia, por ello se deben 
realizar acciones de pre-procesamiento con el objetivo de poder realizar un análisis correcto.\cite{datavstextmining}.

La minería de texto tiene multitud de aplicaciones y puede ser empleada en diferentes 
campos, no solo científicos. Algunas de las aplicaciones que se destacan son: búsqueda de 
información, reconocimiento de texto, clustering, clasificación, análisis de sentimientos… Estas 
aplicaciones se emplean a diario en grandes corporaciones como Amazon o Google para conocer nuestra 
opinión sin tener que ni siquiera que hacer uso de cuestionarios.


\section{Objetivos y campo de aplicación}
\label{sec:objetivos-y-campo}

El objetivo fundamental del trabajo es la comprensión y el aprendizaje de las diferentes
herramientas que permiten realizar un exhaustivo análisis sobre cualquier texto, ya sean cartas, 
artículos periodísticos, discursos transcritos... En este caso se analizarán publicaciones de diferentes 
usuarios en redes sociales con el objetivo de determinar si la minería de textos aplicada en dicho contexto 
funciona como método de análisis.

Como objetivos adicionales se pretende:
\begin{enumerate}

\item Describir y explicar las técnicas utilizadas tanto para la extracción de datos en cualquier texto, 
como para el análisis en cuestión, asi como técnicas adicionales existentes para el mismo proceso.

\item Realizar un caso de estudio en particular aplicando Text Mining y  obtener la mayor cantidad 
de información posible de un texto cualquiera. Dicho caso de estudio contendrá varias etapas 
de desarrollo, en las que se deberán realizar técnicas como análisis exploratorios, limpieza de datos,
análisis de sentimientos…

\item Por último se pretende realizar un estudio que muestre y compare los datos obtenidos durante 
el proceso análitico. 

\end{enumerate}


\section{Descripcion del trabajo}
\label{sec:descripcion}
Como se ha determinado en el apartado anterior, el objetivo fundamental del trabajo es el análisis de 
cualquier tipo de texto, asi como la exposición exhaustiva de los diferentes algoritmos empleados en 
el proceso y su funcionamiento correspondiente. 

Como herramienta de prueba se ha decidido emplear publicaciones de diferentes usuarios en una determina 
red social, en concreto Twitter. Esta red social, al ser una plataforma que permite a multitud de usuarios de 
todo el mundo compartir opiniones o sentimientos sobre ciertos temas, parece un buen lugar donde poder 
realizar el análisis. Las publicaciones a analizar en concreto se denominan tweets y permiten a los usuarios 
expresarse con un máximo de 280 caracteres.

Como herramienta de análisis se pretende emplear el lenguaje de programación R, con el objetivo de disponer 
de una mayor flexibilidad en el análisis comparado con una aplicación software independiente. Si bien existen
otras herramientas de programación como Python que dominan a la perfección este ámbito, R contiene librerías 
que facilitan y extienden capacidades como herramienta de análisis de texto.

\begin{figure}[tphb]
  \centering
  \includegraphics[width=4in]{text-mining-diagram.png}
  \caption{Minería de textos en redes sociales, esquema genérico.}
  \label{img:data-mining-process}
\end{figure}

Como se puede observar en la Figura \ref{img:data-mining-process}, se muestra un esquema sobre el proceso 
general que se seguirá en  cuanto al desarrollo del caso de estudio. Se determinan las etapas y procesos que se 
realizaran hasta conseguir los datos deseados.\cite{esquema-text-mining}

Como ocurre en muchas redes sociales, Twitter pone a disposición de los usuarios una API \cite{api-twitter} que
permite extraer información de la propia aplicación. A diferencia del resto de redes sociales Twitter no solo provee a los usuarios 
de una web services API, sino que permite la posibilidad de emplear librerías como rweet o twitteR que son capaces de 
comunicarse con dicha API.

\section{Metodología y plan de trabajo}
\label{sec:metodologia-y-plan}
En consecuencia de los objetivos y campo de aplicación determinados en la sección \ref{sec:objetivos-y-campo},
se detallan las fases de desarrollo del trabajo. En la Figura \ref{img:workdown-diagram} se pueden 
apreciar las distintas tareas que compondrán el trabajo, así como las distintas subtareas de los mismos.

\begin{figure}[tphb]
  \centering
  \includegraphics[width=6in]{breakdown-diagram.png}
  \caption{Estructura de descomposición del trabajo.}
  \label{img:workdown-diagram}
\end{figure}

A continuación, se determina la dedicación aproximada a cada una de las fases que componen el proyecto.

\begin{enumerate}
\item Fase de inicialización del trabajo: (0,5 meses):
  
  \begin{itemize}
  \item Determinación del contenido del proyecto.
  \item Análisis de las herramientas necesarias para el desarrollo del trabajo.
  \item Acceso a la API de Twitter.
  \end{itemize}

\item Definición de las herramientas de Text Mining (2 meses):

  \begin{itemize}
  \item Definición de algortimos de Text Minign en R.
  \item Definición de paquetes disponibles en R para el análisis de datos en Twitter.
  \end{itemize}
  
\item Desarrollo del caso de estudio (3 meses):

 \begin{itemize}
  \item Conexión con la API de Twitter.
  \item Extracción de datos de Twitter.
  \item Carga de datos.
  \item Limpieza de datos y tokenización.
  \item Análisis exploratorio.
  \item Análisis de sentimientos.
  \end{itemize}

\item Análisis de los datos obtenidos (1 mes):

  \begin{itemize}
  \item Comparación y muestra de los datos obtenidos.
  \item Elaboración de un estudio.
  \end{itemize}

\end{enumerate}
En cada una de estas acciones no solo se deberá tener en cuenta la fase de procesamiento, sino que también se deberá tener 
en cuenta la correspondiente documentación del proceso realizado asi como las diferentes consultas bibliográficas que se crean 
oportunas.

\section{Medios}
\label{sec:medios}

Se determinan a continuación los medios necesarios para el desarrollo del trabajo, así como las 
herramientas que lo complementan.

En primer lugar, se empleará el lenguaje de programación R para el desarrollo del TFG así como la 
aplicación de paquetes y algoritmos dentro del mismo para realizar un correcto minado de los textos.\cite{r}.
Se pretende emplear RGui, aunque es posible emplear herramientas de código abierto como RStudio que permiten escribir y ejecutar 
programas en R.\cite{rstudio}.

Se empleará Twitter como una herramienta de prueba, por ello se debe tener acceso a la API. Para que 
Twitter conceda acceso a su API es necesario crearse una cuenta Twitter Apps y justificar el uso que se 
hará con dicha herramienta.\cite{api-twitter}

El trabajo se escribirá en LaTex, por ello será necesario la instalación de dicho compilador de textos, 
concretamente en su versión para Windows.\cite{miktext}. Además, como se pretende incluir fragmentos de código R en el documento, 
se deberá utilizar el componente Sweave. \cite{Sweave}.






% 
% END Normal sections. Edit/modify all within this section
%%%%%%%%%%%%%%%%%%%%%%%%%%%%%%%%%%%%%%%%%%%%%%%%%%%%%%%%%%%%%%%%%%%%%%%%%%% 
%%%%%%%%%%%%%%%%%%%%%%%%%%%%%%%%%%%%%%%%%%%%%%%%%%%%%%%%%%%%%%%%%%%%%%%%%%% 
%%%%%%%%%%%%%%%%%%%%%%%%%%%%%%%%%%%%%%%%%%%%%%%%%%%%%%%%%%%%%%%%%%%%%%%%%%% 
%%%%%%%%%%%%%%%%%%%%%%%%%%%%%%%%%%%%%%%%%%%%%%%%%%%%%%%%%%%%%%%%%%%%%%%%%%% 
%%%%%%%%%%%%%%%%%%%%%%%%%%%%%%%%%%%%%%%%%%%%%%%%%%%%%%%%%%%%%%%%%%%%%%%%%%% 
%%%%%%%%%%%%%%%%%%%%%%%%%%%%%%%%%%%%%%%%%%%%%%%%%%%%%%%%%%%%%%%%%%%%%%%%%%% 
%%%%%%%%%%%%%%%%%%%%%%%%%%%%%%%%%%%%%%%%%%%%%%%%%%%%%%%%%%%%%%%%%%%%%%%%%%% 


%%%%%%%%%%%%%%%%%%%%%%%%%%%%%%%%%%%%%%%%%%%%%%%%%%%%%%%%%%%%%%%%%%%%%%%%%%% 
% Bibliography
%%%%%%%%%%%%%%%%%%%%%%%%%%%%%%%%%%%%%%%%%%%%%%%%%%%%%%%%%%%%%%%%%%%%%%%%%%% 

%\bibliographystyle{plainnat}
%\bibliographystyle{dinat}
%\bibliographystyle{unsrt}
\bibliographystyle{IEEEtran}

% The following is overly complicated because I was not able to do so in
% another way. The problem is the bibliography command being "called"
% from both the root and anteproyecto directories...
%
% Here define as many bibfiles as needed
\newcommand{\mybibfileOne}{biblio/biblio}
\newcommand{\mybibfileTwo}{biblio/biblio2}
%...
%\newcommand{\mybibfileN}{biblio/biblioN}

% This is for a single bib file
\newcommand{\mybibfiles}{\myreferencespath\mybibfileOne}
% but do this for multiple files
%\newcommand{\mybibfiles}{\myreferencespath\mybibfile1,\myreferencespath\mybibfile2,...,\myreferencespath\mybibfileN}

% Do not touch this
\inputencoding{latin1}
\bibliography{\mybibfiles}
\inputencoding{utf8}

%%% Local Variables:
%%% TeX-master: "../book"
%%% End:


               % EDIT this file if required



\end{document}

