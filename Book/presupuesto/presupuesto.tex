\chapter{Conclusiones y futuras l\'ineas de trabajo}
\label{cha:conclusiones y futuras lineas de trabajo}

Una vez realizado el proceso de evaluación, es posible determinar una serie de conclusiones firmes sobre
los resultados obtenidos:

\begin{itemize}

    \item Aquellas librerías o paquetes cuya heurística está más desarrollada presentan unos resultados
    mejores que herramientas menos cuidadas. El simple uso de expresiones \emph{XPath} dificulta un proceso
    de extracción de calidad.
    \item La calidad de la extracción, no dependen del analizador ni del tipo de lenguaje. La heurística
    y optimización del mismo son los únicos aspectos relevantes en este sentido.
    \item Además de la heurística, el objetivo que se quiera conseguir con el paquete es muy importante 
    para obtener buenos resultados. Paquetes como \textbf{rvest} demuestran que el objetivo de algunas 
    herramientas es simplemente ser capaces de acceder a ciertas etiquetas de documentos HTML.
    \item \textbf{Trafilatura} es el paquete cuya extracción resultante presenta una similitud más cercana
    a lo que un usuario vería en el documento HTML convencional.
    \item El uso de \emph{web scraping} facilita y optimiza el uso tradicional de extracción y almacenado
    de información. Conseguir extraer el 94\% del contenido principal de 101 documentos HTML en tan solo 
    4.4590 segundos, hace que el empleo de herramientas similares a \textbf{Trafilatura} sean buenas 
    opciones para sustituir la extracción tradicional.

\end{itemize}

Como futuras líneas de trabajo a seguir, podrían añadirse nuevas librerías o paquetes de minado web. Sería
interesante la ampliación del proyecto a otros lenguajes de programación como Go, NodeJS, JavaScript...
Además, esto permitiría descubrir nuevas heurísticas que sean comparadas a las ya analizadas.

Otra posible mejora que interesante a introducir, sería la de ampliar la evaluación hacia nuevos idiomas 
en los documentos de prueba. Hasta ahora, las herramientas han sido evaluadas sobre documentos HTML escritos 
en inglés. Podría ser un aspecto positivo para la integridad de la evaluación, evaluar librerías de minado 
que extraigan texto de documentos escritos en diferentes idiomas.