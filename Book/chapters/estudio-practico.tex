
\chapter{Ejemplo práctico: Text Mining con R}
\label{cha:Text-mining-r}

En este capítulo se va a desarrollar un caso práctico sobre minería de textos, con el fin de corroborar 
su utilidad y su importancia en el mundo actual. Para dicho análisis necesitaremos una fuente de 
información de donde extraer los textos a analizar, dicha fuente bien podría ser un artículo de opinión, 
discursos transcritos… Durante este capítulo, se procederá a realizar el proceso de minería de textos 
sobre publicaciones en redes sociales, en este caso Twitter.

Twitter es actualmente una dinámica e ingente fuente de contenidos que, dada su popularidad e impacto, 
se ha convertido en la principal fuente de información para estudios de Social Media Analytics. Multitud 
de empresas emplean este tipo de técnica para obtener información muy valiosa de individuos o de 
corporaciones. Análisis de reputación de empresas, productos o personalidades, estudios de impacto 
relacionados con marketing, extracción de opiniones y predicción de tendencias son sólo algunos ejemplos 
de aplicaciones.

Además de R existen otros lenguajes de programación como Python, MatLab u Octave. Si bien Python es el 
lenguaje que domina en este ámbito, este análisis se realizará a través de la programación en R, pues 
contiene tanto librerías, como paquetes que facilitan y extienden sus capacidades como herramienta de 
análisis de texto.


\section{Introducción}
\label{sec:introduccion}

Tal y como ocurre en muchas redes sociales, Twitter otorga la posibilidad de compartir sus datos tanto 
con empresas como con desarrolladores y/o usuarios particulares. Aunque en la mayoría de casos se trata 
de web Services API, con frecuencia existen librerías que permiten interactuar con la API desde diversos 
lenguajes de programación. 

La forma en la que Twitter permite acceder a su contenido es a través de lo que se conoce como Twitter 
App, al crear dicha Twitter App, se adquieren una serie de claves y tokens de identificación que 
permiten acceder a la aplicación y consultar la información necesaria.  

Algo que se deberá tener en cuenta durante este análisis, es que Twitter tiene una normativa que regula 
la frecuencia máxima de peticiones, así como la cantidad máxima de tweets que se pueden extraer. Durante 
la fase de extracción de la información, se deberá tener en cuenta dichos limites con el objetivo de 
evitar ser sancionado por la organización.

\section{Acceso a la API}
\label{sec:acceso-api}

Para acceder a la API de Twitter, como se indica en la documentación de la misma, existen dos métodos de 
acceso Oauth2 y Oauth1a. El acceso con cada uno de ellos dependerá del tipo de información que se desee 
extraer \cite{oauth1a}. Como veremos a continuación, nosotros nos centraremos en Oauth1a.

\begin{figure}[tphb]
  		   \centering
     		   \includegraphics[width=4.5in]{Oauth1a.png}
  		   \caption{Funcionamiento de Oauth1a}
  		   \label{img:oauth1a}
\end{figure}

Puesto que se desea extraer información específica de algunos usuarios de la plataforma, el metodo de 
acceso será a través de Oauth1a. Así pues, OAuth1a para funcionar requiere 
de cuatro elementos que provienen de la aplicación de Twitter creada por el usuario desarrollador, estos elementos son:

\begin{enumerate}
\item Clave del consumidor
\item Clave del consumidor secreta
\item Token de acceso
\item Token de acceso secreto
\end{enumerate}


Todos estos recursos deberán ser proporcionados por la plataforma de Twitter para desarrolladores. Una 
vez se disponen de dichos recursos, realizamos el proceso de identificación y obtención de tokens \cite{error_oauth1a}, 
no sin antes cargar las librerías necesarias para el desarrollo de dicho caso práctico.


\begin{Schunk}
\begin{Sinput}
>   #En primer lugar se cargan las librerias necesarias.
>   library(rtweet)
>   library(tidyverse)
>   library(knitr)
\end{Sinput}
\end{Schunk}






Ahora que las librerías necesarias están cargadas, se procede a crear las variables que almacenarán los 
recursos de acceso. A conitnuación, solo debemos llamar a la función de acceso. Esta funcion, se encargará de enviar 
una solicitud para generar tokens OAuth 1.0, dichos tokens solo se podrán utilizar para solicitar información \cite{create_token}.


\begin{Schunk}
\begin{Sinput}
>   #Ahora se definen ciertas variables que almacenaran la informacion de acceso
>   nombre_app <- "MarquezDavidTFG"
>   clave <- "bsS2cbbZe7BDsxFPLYRM0GKJ8"
>   clave_secreta <- "WvUilwEZbgJhHiUgpjyboQcQHYSSCNKwImHF1TeINe9aWskkDo"
>   token_acceso <- "3147132436-kFG9XkuWsdI8n1KPVZQTOrf6rw45lrqPEPoUXPr"
>   token_acceso_secreto <- "zK8dyBAgB4sTrRhjVBXmNCAjX4QQvCUfQIXckQmTrcAix"
>   #Creamos nuestro twitter token
>   twitter_token <- create_token(app=nombre_app, consumer_key=clave, consumer_secret=clave_secreta, access_token=token_acceso, access_secret=token_acceso_secreto)
\end{Sinput}
\end{Schunk}









