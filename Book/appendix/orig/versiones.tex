
\chapter{Versiones}
\label{cha:versiones}

En este apartado incluyo el historial de cambios más relevantes de la
plantilla a lo largo del tiempo.

No empecé este apéndice hasta principios de 2015, con lo que se ha
perdido parte de la información de los cambios importantes que ha ido
sufriendo esta plantilla.


\begin{itemize}

\item Abril 2015:
  \begin{itemize}
  \item Ahora manejamos masculino/femenino en algunos sitios (el/la,
    autor/autora, alumno/alumna, del/de la, ...). Hay que definir
    variable con el sexo del autor (todavía queda pendiente lo de los
    tutores y tal). NOT FINISHED!!
  \end{itemize}

\item Abril 2015:
  \begin{itemize}
  \item Comienzo de intento de hacer un \texttt{make bare} para que deje
    los capítulos mondos y lirondos. Afecta a la creación de ficheros
    \texttt{*-bare.tex} en los directorios \texttt{./chapters} y
    \texttt{./appendix}. NOT FINISHED!!
  \end{itemize}

\item Enero 2015:
  \begin{itemize}
  \item Solucionado el problema (gordo) de compilación del
    \texttt{anteproyecto.tex} y el \texttt{book.tex}, debido al uso de
    paths distintos en la compilación de la bibliografía. El sistema se ha
    complicado un poco (ver
    \texttt{biblio\textbackslash{}bibliography.tex}).
  \item Añadido un (rudimentario) sistema para generar pdf con las
    diferencias entre el documento en su estado actual y lo último
    disponible en el repositorio (usando \texttt{latexdiff}).
  \end{itemize}
\item Diciembre 2015:
  \begin{itemize}
  \item Separada la compilación del anteproyecto de la del documento
    principal. Para el primero se ha creado el directorio
    \texttt{anteproyecto} donde está todo lo necesario.
  \end{itemize}
\end{itemize}

%%% Local Variables:
%%% TeX-master: "../book"ve
%%% End:


